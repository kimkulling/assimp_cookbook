\begin{document}
\section{Chapter 1: Install Asset-Importer-Lib}
If you want to work with the Asset-Importer-Lib you first have to install the package onto your development-system. There are different options you can choose of for installing it.
If you are interested in using the library directly you can install it from the last pre-build binaries, which you can download, for your platform. If you want to test the
latest features and bugfixes you can directly check-out the latest source and build Asset-Importer-Lib from them. 
In the next sections I will describe each option.

\subsection{Install Asset-Importer-Lib from installer ( Windows ) }
If you are working on Windows you can install the pre-build binaries from the http://www.assimp.org. Just naviagte to assimp.org, follow the link to the latest release and download the 
provided MSI to install the latest release.

\subsection{Install Asset-Importer-Lib from installer ( Linux-Ubuntu ) }
If you are working on Ubutu and you want to install Asset-IMporter-Lib you can install the DPKG-package via apt-get:
Just refresh your installation in a fresh opened console:
> sudo apt-get update
> sudo apt-get install assimp-dev
\par
The package-manager will install the developer-libs and the corresponding includes for your Ubuntu-installation. It will also resolve all needed packages with 
the right version and install them as well.
\subsection{Build Asset-Importer-Lib from source}
If you want to hack around it makes sense to get the current developer branch from the project-page located on GitHub:
\begin{lstlisting}[label=some-code,caption=Checkout the source from the assimp github project repo]
> git clone https://github.com/assimp/assimp.git
\end{lstlisting}

To build Asset-Importer-Lib from the source you first need to install cmake. It can be found at https://cmake.org/ . CMake is a tool to generate the build environment 
for your preferred working environment. It can be used to generate a workspace for eclipse, QTCreator or for Visual-Studio. Of course it is also useful for generating just 
a makefile for GNU-make or ninja or something else. Just check its website if you want to learn more about its features. After installing it open a console and type
\begin{lstlisting}[label=some-code,caption=Generate the makefile for Linux]
> cd assimp
> cmake CMakefiles.txt
\end{lstlisting}

CMake will generate a working build environment for the Asset-Importer-Lib. Now you can open the Visual-Studio-Solution or just start a GNU-Make run on your developer PC. 
If you want to generate a special build-environment you can use the option -G <Build-Tool> . If you want to ee which gernerators are supported for your system just type 
\begin{lstlisting}[label=some-code,caption=Run the build on Linux using 4 cores]
> cmake -h
\end{lstlisting}

For instance for Linux you can use the following command:
\begin{lstlisting}[label=some-code,caption=Run the build on Linux using 4 cores]
> make -j4
> make install
\end{lstlisting}

The option j defines the recommendation how much core are free for the build of Asset-Importer-Lib.
After building the library with success you will bin a folder called bin containing all executables and a folder called lib containing all libraries. Asset-Importer-Lib needs
some third-party-libs to get build successfule. Most of them are part of the repository. The CMake-run will check whether any of these libs are already installed on your working 
environment. If not the version from the repository will be used instead. So it is possible to build the lib without installing anything else.
\par
For Windows you can open the generated Visual-Studio-Project or just generate a nmake makefile.
\subsection{What is in the installation}
After installing it you will have a command-lib-tool called assimp, which can be used to get information about an asset like how many vertices are part of the model or you 
can convert it to a supported experter-format like obj to Collada form instance.
\par
The next thing the Asset-IMporter-Lib provides are the developer-library with the corresponding include files. You can find them at the following places:
\begin{center}
\begin{tabular}{ | l | c | c | }
\hline
Filetype & Windows & Linux / MacOS \\
\hline
Includes                       & ?       & ?             \\
\hline
Librarys ( *dll, *lib / *so )  & ?       &               \\
\hline

\end{tabular}
\end{center}

\end{document}

\begin{document}
\section{Chapter 1: Install Asset-Importer-Lib}
If you want to use the Asset-Importer-Lib you first have to install the package for your development-environment. There are different options you can choose of you want to
work with Asset-Importer-Lib. If you are interested in using the library you can install it from the last pre-build binaries for your platform. If you want to test the
latest features and bugfixes you can directly check-out the latest sources and build Asset-Importer-Lib from them.
In the next sections I will describe each option more detailed.

\subsection{Install Asset-Importer-Lib from installer ( Windows ) }
If you are working on Windows you can install the pre-build binaries from the web. Just open assimp.org, follow the link to the latest release and download the 
latest MSI to install the latest release.

\subsection{Install Asset-Importer-Lib from installer ( Linux-Ubuntu ) }
If you are working on Ubutu and you want to install Asset-IMporter-Lib you can install the DPKG-package via apt-get:
Just refresh your installation in a fresh opened console:
> sudo apt-get update
> sudo apt-get install assimp-dev

The package-manager will install the developer-libs and the corresponsding includes on your Ubuntu-installation.
 
\subsection{Build Asset-Importer-Lib from source}
If you want to hack around it makes sense to get the current developer branch from the project-page located on GitHub:
> git clone https://github.com/assimp/assimp.git

To build Asset-Importer-Lib from the source you need to install cmake first. It can be found at https://cmake.org/ . CMake is a tool to generate a build environment 
for your preferred working environment. It can be used to generate a workspace for eclipe, QTCreator for Visual-Studio. Of course it is also useful for generating just 
a makefile for GNU-make or ninja or something else. Just check its website if you want to learn more about its features. After installing it open a console and type
> cd assimp
> cmake CMakefiles.txt

CMake will generate a working build environment for the Asset-Importer-Lib. Now you can open the Visual-Studio-Solution or just start a GNU-Make run on your dveloper PC.

\end{document}